\chapter{Fazit}

In diesem Kapitel werden die zuvor gestellten Fragen beantwortet. Darüber hinaus wird die potenzielle Anwendung des Prototyps diskutiert.

\section{Beantwortung der Forschungsfrage}

Folgende Fragen wurden am Anfang gestellt:

\begin{itemize}
	\item Welches Speichersystem ist im Hinblick auf Kosten, Performance und Verfügbarkeit für die Persistenz von Binärdaten besonders geeignet? 
	\item Wie können Daten durch sichere, zeitlich begrenzte URLs bereitgestellt werden?
\end{itemize}

Um die erste Frage zu beantworten, werden die Punkte des theoretischen Teils aufgegriffen. Es wurden verschiedene Speicherarten wie Objekt-, Block-, und File Storage untersucht. Dabei stellte sich heraus, dass Objekt Storage als Speichersystem für die Anforderungen von leoticket geeignet ist. Einige Cloud Provider stellen Objekt Storage zur Speicherung von Daten zur Verfügung und sind im Markt stark vertreten. Es wurden die zwei größten Cloud Provider AWS und GCP betrachtet und eine Vergleichsbasis hergestellt im Hinblick auf Kosten, Performance, Verfügbarkeit, Sicherheit, Bereitstellung der Daten und API Anbindungen. Bei der sicheren Speicherung war es wichtig, dass die Daten vertraulich gespeichert werden und nur Berechtigte Zugriff auf sie haben.\\

 Datenverschlüsselungsmethoden wurden bei beiden Cloud Providern betrachtet und dabei festgestellt, dass die SSE C zwar die stärkste unabhängige Sicherheit bietet, jedoch das Risiko besteht, selbstverwaltete und gespeicherte Schlüssel zu verlieren. Außerdem müssten Mitarbeiter dafür geschult werden, was extra Aufwand bedeutet. Aus diesem Grund wurde für die Implementierung des Prototyps die SSE-KMS customer-managed Methode verwendet, damit der Nutzer die Schlüssel in der KMS von den Providern selbst erstellen und verwalten kann. So bleibt die Kontrolle erhalten und verschafft höhere Sicherheit. Je nach den gewählten Speicherklassen versprechen beide Anbieter eine Verfügbarkeit von mindestens 99.5\%. Bei der Untersuchung der Speicherklassen in Verbindung mit Kosten und Performance stellte sich heraus, dass die Standard-IA von AWS und die Nearline von GC besser zu den Anforderungen von leoticket passen. Da die Latenz für leoticket kein Kriterium darstellt und vernachlässigt werden kann, fallen die Standard Klassen beider Provider weg. Die Standardklassen bieten zwar eine leistungsfähige Speicherlösung, jedoch gehen die eingesetzten Mittel über das hinaus, was tatsächlich benötigt wird. Da Daten über einen Zeitraum von bis zu zehn Jahren gespeichert werden müssen, sind die Speicherkosten für die Standardklassen im Vergleich zu anderen Optionen zu hoch. Die Speicherung der Daten auf längerer Sicht steht mehr im Fokus, da auf eine Datei im Durschnitt nur zweimal zugegriffen wird. Die OneZone-IA fällt ebenfalls weg, da die Daten nur in einer Availability Zone gespeichert werden. Das Risiko ergibt sich durch Nicht-Verfügbarkeit der Daten durch Speicherung in lediglich einer Zone. Daten müssen auf Abruf schnell zugreifbar sein, deshalb sind die OneZone-IA und die Coldline nicht geeignet, da sie eher für selten abgerufen Daten angepasst sind. Die Abrufkosten dieser Speicherklassen sind am höchsten und nicht zu empfehlen.\\

Insgesamt wird für die Persistenz von Binärdaten ein Object Storage mit den Speicherklassen Standard-IA von AWS und Nearline von GC empfohlen. Sie bieten die nötigen Funktionen an, um die Anforderungen zu decken und kostengünstig Daten für längere Zeit zu speichern und dabei eine hohe Verfügbarkeit und Performance zu bieten. Die Entscheidung hängt auch von den persönlichen Präferenzen des Unternehmens ab. Beide Provider bieten eine gute Objektspeicherung kostengünstig und leistungsfähig an.\\

Bei der zweiten gestellten Frage geht es um die Bereitstellung der Daten durch signierte zeitlich begrenzte URLs. Diese Frage wurde durch Vergleichen der SDKs bei der Implementierung des Prototypen untersucht und bewertet. Es stellt sich heraus, dass beide Cloud Provider die Funktionen anbieten, signierte URLs zu erstellen und zeitlich begrenzt bereitzustellen. Mithilfe des Prototypen kann man Dateien hochladen und sie durch URLs bereitstellen. Durch Klicken auf den generierten Link werden die Daten heruntergeladen. So kann verhindert werden, Dateien direkt in Email Anhängen hinzuzufügen, sondern durch Links bereitzustellen. Diese Dateien werden von den Buckets entschlüsselt heruntergeladen und Nutzer können ohne AWS oder GC Credentials darauf zugreifen. Über den Prototypen kann man die Minuten, in denen der Link valide ist, angeben. GC und AWS stellen ausführliche Dokumentationen auf den offiziellen Seiten bereit, um diese Funktionen zu implementieren und in verschiedenen Programmiersprachen anzuwenden.\\

So kann leoticket vom alten System zu der neuen empfohlenen Speicherlösung wechseln, um Daten sicher und schnell bereitzustellen und für längere Zeit zu speichern. Es ist zu beachten, dass diese Arbeit lediglich dem Vergleich und der Veranschaulichung beider Cloud Provider dient und dass jedes Unternehmen unterschiedliche Anforderungen aufweist. Diese Arbeit dient als Stütze und zum Testen der Technologien auf Basis des Prototypen. 

\section{Potenzielle Anwendung des Prototyps}

Der Prototyp wurde entwickelt, um sich mit den Technologien der Cloud Provider auseinanderzusetzen. Durch die Anwendung konnten Performance Analysen durchgeführt werden, die zur Auswahl des Speichersystems beitragen. Die Anwendung kann ausgebaut werden, sodass sie den Bedürfnissen der Unternehmen entspricht. Sie stellt eine Bibliothek dar, die in eigene Anwendungen integriert werden kann. Um sich mit den Technologien vertraut zu machen, können Dateien hoch- und heruntergeladen werden. Außerdem ist es möglich Terraform Buckets mit den benötigten Einstellungen und Rechten zu erstellen ohne die Cloud Konsole verwenden zu müssen.\\

Der Prototyp wurde dabei an die Anforderungen von leoticket angelehnt und angepasst, damit er in das Produkt integriert werden kann. Er dient zur Hilfestellung für das Wechseln von Galera Cluster in ein neues Objekt Storage System für Binärdaten. 
