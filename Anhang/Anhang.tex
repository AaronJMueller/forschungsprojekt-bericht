% Anhang

\chapter{Anhang}

\section{Repositories}
\subsection{Github Link}

\subsubsection*{Prototyp:}

HTTPS: \url{https://github.com/gmzbae/cloud-gcs-aws3.git}

\begin{verbatim}SSH: git clone git@github.com:gmzbae/cloud-gcs-aws3.git\end{verbatim}			

\subsubsection*{Thesis:}

HTTPS: \url{https://github.com/gmzbae/bachelor-thesis-latex.git}

\begin{verbatim}SSH: git clone git@github.com:gmzbae/bachelor-thesis-latex.git \end{verbatim}

\newpage

\subsection{Code Snippets}

\begin{lstlisting}[language=XML, caption=pom.xml - Abhängigkeiten des Prototyps]
	<?xml version="1.0" encoding="UTF-8"?>
<project xmlns="http://maven.apache.org/POM/4.0.0" xmlns:xsi="http://www.w3.org/2001/XMLSchema-instance"
         xsi:schemaLocation="http://maven.apache.org/POM/4.0.0 https://maven.apache.org/xsd/maven-4.0.0.xsd">
    <modelVersion>4.0.0</modelVersion>
    <parent>
        <groupId>org.springframework.boot</groupId>
        <artifactId>spring-boot-starter-parent</artifactId>
        <version>3.0.7</version>
        <relativePath/>
    </parent>
    <groupId>de.leomedia</groupId>
    <artifactId>cloud_gcStorage_AwsS3</artifactId>
    <version>0.0.1-SNAPSHOT</version>
    <name>cloud_gcStorage_AwsS3</name>
    <description>cloud_gcStorage_AwsS3</description>
    <properties>
        <java.version>17</java.version>
        <spring-cloud-gcp.version>4.3.1</spring-cloud-gcp.version>
        <spring-cloud.version>2022.0.3</spring-cloud.version>
    </properties>
    <dependencies>
        <dependency>
            <groupId>com.google.cloud</groupId>
            <artifactId>spring-cloud-gcp-starter-storage</artifactId>
        </dependency>

        <dependency>
            <groupId>io.awspring.cloud</groupId>
            <artifactId>spring-cloud-aws-s3</artifactId>
            <version>3.0.0</version>
        </dependency>

        <dependency>
            <groupId>org.springframework.boot</groupId>
            <artifactId>spring-boot-starter-test</artifactId>
            <scope>test</scope>
        </dependency>
    </dependencies>
    <dependencyManagement>
        <dependencies>
            <dependency>
                <groupId>org.springframework.cloud</groupId>
                <artifactId>spring-cloud-dependencies</artifactId>
                <version>${spring-cloud.version}</version>
                <type>pom</type>
                <scope>import</scope>
            </dependency>
            <dependency>
                <groupId>com.google.cloud</groupId>
                <artifactId>spring-cloud-gcp-dependencies</artifactId>
                <version>${spring-cloud-gcp.version}</version>
                <type>pom</type>
                <scope>import</scope>
            </dependency>
        </dependencies>
    </dependencyManagement>

    <build>
        <plugins>
            <plugin>
                <groupId>org.springframework.boot</groupId>
                <artifactId>spring-boot-maven-plugin</artifactId>
            </plugin>
        </plugins>
    </build>

</project>

\end{lstlisting}

\begin{lstlisting}[caption=application.properties Datei - Umgebungsvariablen des Prototyps]
CLOUD_PROVIDER=${cloud_provider}
PROJECT_ID=${GOOGLE_PROJECT_ID}
BUCKET_NAME=${bucket_name}
GC_JSON_KEY_PATH=${GOOGLE_SERVICE_ACCOUNT_FILEPATH}
ENCRYPTION_KEY=${sse_kms_key_id_arn}
STORAGE_CLASS=${storage_class}

TEST_BUCKET_NAME=${test_bucket_name}
TEST_FILES_PATH=${test_files_path}
TEST_OBJECT_KEY_PATTERN=${test_object_key_pattern}
TEST_FILE_COUNT=${test_file_count}
TEST_GC_ENCRYPTION_KEY=${test_gc_encryption_key}
TEST_AWS_ENCRYPTION_KEY=${test_aws_encryption_key}
\end{lstlisting}

\begin{lstlisting}[language=Java, caption=getCloudStorageService() Methode der Klasse CloudStorageServiceFactory]
public static CloudStorageService getCloudStorageService(String cloudProvider, String projectId, String jsonKeyPath, S3Presigner presigner, S3Client s3Client) throws IOException {
     if ("AWS".equalsIgnoreCase(cloudProvider)) {
         return new AWSS3StorageService(s3Client, presigner);
     } else if ("Google Cloud".equalsIgnoreCase(cloudProvider)) {
         return new GCStorageService(projectId, jsonKeyPath);
     } else {
         throw new IllegalArgumentException("Invalid cloud provider: " + cloudProvider);
     }
 }	
\end{lstlisting}

\newpage

\begin{lstlisting}[language=Java, caption=performanceRunner() Methode der Klasse PerformanceTest]
/**
* This test method calls the given methods and starts the performance runner.
*
*/
@Test
public void performanceRunner() throws IOException {

    generateFiles();

    calculateUploadObjectsToS3_TimeMeasure();
    calculateUploadObjectsToCloudStorage_TimeMeasure();

    calculateDownloadObjectsFromS3_TimeMeasure();
    calculateDownloadObjectsFromCloudStorage_TimeMeasure();

}
\end{lstlisting}

\begin{lstlisting}[language=Java, caption=calculateUploadObjectsToS3\_TimeMeasure() Methode der Klasse PerformanceTest]
/**
 * This method uploads multiple objects to S3
 * and measures the current time it took in milliseconds.
 */
public void calculateUploadObjectsToS3_TimeMeasure() {

    AWSS3StorageService awss3StorageService = new AWSS3StorageService(s3Client, S3Presigner.builder().build());

    long startTime = System.currentTimeMillis();

    for(int i = 1; i <= fileCount; i++){
        awss3StorageService.uploadObject(bucketName, key + i + ".txt", filePath + key + i + ".txt", aws_encryption_key, storageClass);
    }

    long endTime = System.currentTimeMillis();
    long elapsedTime = endTime - startTime;

    logger.info("Elapsed Time for " + fileCount + "  Object Uploads in S3: " + elapsedTime +  " ms.");

}	
\end{lstlisting}

\newpage

\begin{lstlisting}[language=Java, caption=calculateDownloadObjectsFromS3\_TimeMeasure() Methode der Klasse PerformanceTest]
/**
 * This method downloads multiple objects from S3
 * and measures the current time it took in milliseconds.
 */
public void calculateDownloadObjectsFromS3_TimeMeasure(){

    AWSS3StorageService awss3StorageService = new AWSS3StorageService(s3Client, S3Presigner.builder().build());

    long startTime = System.currentTimeMillis();

    for(int i = 1; i <= fileCount; i++){
        awss3StorageService.getPresignedUrl(bucketName, key + i + ".txt", 60, aws_encryption_key);
    }

    long endTime = System.currentTimeMillis();
    long elapsedTime = endTime - startTime;

    logger.info("Elapsed Time for " + fileCount + " Object Downloads in S3: " + elapsedTime +  " ms.");

}
\end{lstlisting}


\begin{lstlisting}[language=Java, caption=calculateUploadObjectsToCloudStorage\_TimeMeasure() Methode der Klasse PerformanceTest]
/**
 * This method uploads multiple objects to Cloud Storage
 * and measures the current time it took in milliseconds.
 */
public void calculateUploadObjectsToCloudStorage_TimeMeasure() throws IOException {
    GCStorageService googleCloudStorageService = new GCStorageService(projectId, gcJsonKeyPath);

    long startTime = System.currentTimeMillis();

    for(int i = 1; i <= fileCount; i++){
        googleCloudStorageService.uploadObject(bucketName, key + i + ".txt", filePath + key + i + ".txt", gc_encryption_key, storageClass);
    }

    long endTime = System.currentTimeMillis();
    long elapsedTime = endTime - startTime;

    logger.info("Elapsed Time for " + fileCount + " Object Uploads in Cloud Storage: " + elapsedTime +  " ms.");

}
\end{lstlisting}

\newpage

\begin{lstlisting}[language=Java, caption=calculateDownloadObjectsFromCloudStorage\_TimeMeasure() Methode der Klasse PerformanceTest]
/**
 * This method downloads multiple objects from Cloud Storage
 * and measures the current time it took in milliseconds.
 */
public void calculateDownloadObjectsFromCloudStorage_TimeMeasure() throws IOException {

    GCStorageService googleCloudStorageService = new GCStorageService(projectId, gcJsonKeyPath);

    long startTime = System.currentTimeMillis();

    for(int i = 1; i <= fileCount; i++){
        googleCloudStorageService.getPresignedUrl(bucketName, key + i + ".txt", 60, gc_encryption_key);
    }

    long endTime = System.currentTimeMillis();
    long elapsedTime = endTime - startTime;

    logger.info("Elapsed Time for " + fileCount + " Object Downloads in Cloud Storage: " + elapsedTime +  " ms.");

}
\end{lstlisting}

\begin{lstlisting}[language=Java, caption=generateFiles() Methode der Klasse PerformanceTest]
/**
 * This method writes the generated content into the generated file.
 */
public void generateFiles(){

    for(int i = 1; i <= fileCount; i++){
        String filename = filePath + key + i + ".txt";
        String content = generateContent();

        try {
            FileOutputStream fos = new FileOutputStream(filename);
            fos.write(content.getBytes(StandardCharsets.UTF_8));
            fos.close();
            logger.info("File generated: " + filename);
        } catch (IOException e) {
            logger.error("An error occurred while generating the file: " + e.getMessage());
        }
    }
}
\end{lstlisting}

\newpage

\begin{lstlisting}[language=Java, caption=generateContent() Methode der Klasse PerformanceTest]
/**
 * This method generates random string content in size  of 100kb.
 * @return a random generated string content
 */
public String generateContent() {
    StringBuilder sb = new StringBuilder();
    int contentSize = 100 * 1024; // Convert KB to bytes

    while (sb.length() < contentSize) {
        sb.append("Lorem ipsum dolor sit amet, consectetur adipiscing elit. ");
    }

    logger.info("Generated string content");

    return sb.substring(0, contentSize);
}
\end{lstlisting}

\newpage


\clearpage