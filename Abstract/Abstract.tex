% Kurzfassung, Abstract
\renewcommand\abstractname{Zusammenfassung}

\begin{abstract}
Das Ziel der vorliegenden Arbeit ist die Evaluierung eines geeigneten Systems zur Speicherung und Bereitstellung von Binärdaten im Kontext von Web Services. Diese Binärdaten sollen Kunden von leoticket zur Verfügung gestellt werden. Dabei werden die Anforderungen wie Performance, Verfügbarkeit, Sicherheit und eine Möglichkeit für Integrationen durch APIs in Software Produkten im Beispiel mit leoticket gestellt. Durch die Realisierung eines Prototyps anhand ausgewählter Speicherlösungen werden die Binärdaten durch sichere, zeitlich begrenzte URL's bereitgestellt. Folgende Fragen werden gestellt: Welches Speichersystem ist im Hinblick auf Kosten, Performance und Verfügbarkeit für die Persistenz von Binärdaten besonders geeignet? Wie können Daten durch sichere, zeitlich begrenzte URL’s bereitgestellt werden? Verschiedene Speichersysteme werden verglichen und anhand von Kostenkalkulationen bewertet. Durch die Auswertung des Vergleichs wird der Prototyp implementiert und Messungen auf Testdaten durchgeführt, welches die wissenschaftliche Arbeit stützt. Es stellt sich heraus, dass die Cloud Provider AWS und GC Object Storage Lösungen wie S3 bzw. Cloud Storage anbieten, die für die Speicherung und Bereitstellung von Binärdaten dienen. Durch die Kosten-, und Performance Analyse stellte sich heraus, dass die Speicherklassen Standard-IA von AWS und die Nearline von GC die Anforderungen von leoticket weitestgehend erfüllt. Sie bieten eine hochverfügbare, kostengünstige und leistungsfähige Speicherung an, die durch verschiedene Einstellungen wie die SSE-KMS für die Sicherheit sorgt und durch SDKs eine gute Integration in eigene Anwendungen verspricht. Durch die bereitgestellten Methoden für die Generierung von signierten, zeitlich begrenzten URLs können die Daten an die Ticketkäufer von leoticket ohne weitere Authentifizierung aus der Kundensicht zur Verfügung gestellt werden. Diese Feststellung und Empfehlung dient dem Umstieg auf neue Speicherlösungen mit höherer Performance und Sicherheit mit akzeptablen Kosten.

\end{abstract}

\renewcommand\abstractname{Abstract}      

%TODO Abstract Englisch nochmal übersetzen

\begin{abstract}
The aim of this thesis is to evaluate a suitable system for storing and providing binary data in the context of web services. Requirements such as performance, availability, security, and API integration are set. By implementing a prototype based on the selected storage solution, binary data is provided through secure, time-limited URLs. The following questions are addressed: Which storage system is particularly suitable for persisting binary data in terms of cost, performance, and availability? How can data be provided through secure, time-limited URLs? Different storage systems are compared and evaluated based on cost calculations. By evaluating the comparison, the prototype is implemented and measurements are taken on test data, which supports the scientific work. The result can be used to switch to new storage solutions with higher performance and security at acceptable costs.
\end{abstract}

\clearpage

