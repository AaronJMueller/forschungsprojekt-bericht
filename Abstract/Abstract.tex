% Kurzfassung, Abstract
\renewcommand\abstractname{Zusammenfassung}

\begin{abstract}
Das Ziel der vorliegenden Arbeit ist die Evaluierung eines Systems zur Speicherung und Bereitstellung von Binärdaten im Kontext von Web Services. Diese Binärdaten sollen Kunden des Produkts leoticket der Leomedia GmbH zur Verfügung gestellt werden. Dabei werden Anforderungen wie Performance, Verfügbarkeit, Sicherheit und eine Möglichkeit für Integrationen durch APIs in Software Produkten am Beispiel des Produkts leoticket der Leomedia GmbH dargestellt. Durch die Realisierung eines Prototyps anhand ausgewählter Speicherlösungen werden die Binärdaten durch sichere, zeitlich begrenzte URL's bereitgestellt. Folgende Fragen werden gestellt: Welches Speichersystem ist im Hinblick auf Kosten, Performance und Verfügbarkeit für die Persistenz von Binärdaten besonders geeignet? Wie können Daten durch sichere, zeitlich begrenzte URL’s bereitgestellt werden? Verschiedene Speichersysteme werden verglichen und anhand von Kostenkalkulationen bewertet. Durch die Auswertung des Vergleichs werden der Prototyp implementiert und Messungen auf Testdaten durchgeführt. Die Kosten-, und Performance Analyse ergibt, dass die Speicherklassen Standard-IA von AWS sowie die Nearline von GC die Anforderungen des Produkts leoticket der Leomedia GmbH weitestgehend erfüllen. Sie bieten eine hochverfügbare, kostengünstige und leistungsfähige Speicherung an, die durch verschiedene Einstellungen wie die SSE-KMS Sicherheit bietet und durch SDKs eine gute Integration in eigene Anwendungen verspricht. Durch die bereitgestellten Methoden für die Generierung von signierten, zeitlich begrenzten URLs können die Daten an die Ticketkäufer von leoticket ohne weitere Authentifizierung als Kundensicht zur Verfügung gestellt werden. Diese Feststellung und Empfehlung dient dem Umstieg auf neue Speicherlösungen mit höherer Performance und Sicherheit bei akzeptablen Kosten.

\end{abstract}

\renewcommand\abstractname{Abstract}      

\begin{abstract}
The aim of this thesis is to evaluate a system for storing and providing binary data in the context of web services. Requirements such as performance, availability, security, and API integration being defined. By implementing a prototype based on the selected storage solution, binary data is provided by means of secure, time-limited URLs. The following questions are addressed: Which storage system is particularly suitable for persisting binary data in terms of cost, performance, and availability? How can data be provided through secure, time-limited URLs? Different storage systems are being compared and evaluated providing cost estimations. By evaluating the comparison, the prototype is implemented and measurements are taken on test data. The results suggest a switch to a new storage solution thereby improving both performance and security at acceptable costs.
\end{abstract}

\clearpage

