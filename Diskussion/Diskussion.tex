\chapter{Diskussion}

In diesem Kapitel werden die Kalkulations-, und Messergebnisse der Performance analysiert und interpretiert. Anschließend wird der Prototyp bewertet und dabei auf dessen Grenzen eingegangen.
  
\section{Analyse und Interpretation der Ergebnisse}

Basierend auf der Kostenanalyse und den Ergebnissen der Kalkulation ergibt sich, dass die Speicherklassen OneZone-IA von AWS und Coldline von GC die niedrigsten Kosten für die Speicherung von Daten aufweisen. Jedoch sind Datenabrufe in diesen Speicherklassen als kostenintensiv zu betrachten. In der Coldline-Klasse belaufen sich die Kosten für Klasse A-, und B-Operationen jeweils auf 4,84 Euro, während bei AWS PUT-Anfragen, die mit Klasse A gleichzusetzen sind, 2,50 Euro und GET-Anfragen, die mit Klasse B gleichzusetzen sind, 0,50 Euro betragen. Insgesamt sind die Kosten für die Coldline-Klasse im Vergleich zur OneZone-IA-Klasse jedoch günstiger.\\ 

Die Entscheidung für OneZone-IA als Speicherklasse birgt das Risiko auf die Nicht-Verfügbarkeit von Daten bei Ausfall der AZ, da die Daten nur in einer Verfügbarkeitszone gespeichert werden. Dies steht im Widerspruch zu den Anforderungen von leoticket an eine hohe Verfügbarkeit. Obwohl die Speicherkosten in dieser Klasse am niedrigsten sind, sind die Kosten für Datenabrufe im Vergleich zu anderen Speicherklassen höher. Für leoticket ist es jedoch von Bedeutung, auf Objekte mindestens zweimal zugreifen zu können und sie den Kunden zur Verfügung zu stellen. Bei älteren Daten, die mehrere Jahre zurückliegen, kann es ratsam sein, die Objekte in S3 Glacier Instant Retrieval oder in GC Archive Storage zu verschieben, um Kosten für die Archivierung einzusparen und die Daten in Bereichen von Millisekunden zur Verfügung stehen können.\\

Die Verfügbarkeit der Coldline-Klasse ähnelt der OneZone-IA-Klasse. Die Datenabrufe in dieser Klasse können mehrere Stunden dauern. Sie ist nicht für Szenarien geeignet, die einen sofortigen Zugriff auf Daten erfordern oder in denen häufiger auf die Daten zugegriffen werden muss. Im Gegensatz zur OneZone-IA ist die Coldline jedoch nicht auf eine einzelne Verfügbarkeitszone beschränkt. Aufgrund ihrer geringeren Kosten und der Verfügbarkeit in mehreren Availability Zones ist Coldline die bessere Option für die langfristige Archivierung von Daten. Da die Kosten für Datenabrufe in dieser Klasse höher sind als in allen anderen Speicherklassen, ist sie nicht geeignet für Daten, auf die mehr als einmal zugegriffen werden muss.\\

Im Rahmen der Performance-Analyse zeigte die OneZone-IA eine bessere Leistung als die Coldline. Bereits beim Upload von zehn Dateien war die OneZone-IA deutlich schneller. In diesem Szenario war sie 20.6\% schneller wie die Coldline. Beim Upload von 100 Dateien wurde ein Unterschied von 63\% festgestellt, wobei die Coldline etwa 20 Sekunden und die OneZone-IA nur etwa sieben Sekunden benötigte. Bei 1000 Dateien bewegten sich beide Speicherklassen im Minutenbereich, wobei auch hier die OneZone-IA 55.34\% schneller wie die Coldline war. Beim Download von Dateien bewegten sich beide Klassen in Millisekunden Bereichen. Ab 1000 Dateien war die Coldline ungefähr 55\% langsamer als die OneZone-IA.\\

Die Speicherklassen Standard-IA und Nearline weisen bei den Speicherkosten nur geringfügige Preisunterschiede  auf, die sich auf maximal zwei Euro belaufen. Dabei ist Nearline maximal zwei Euro günstiger als Standard-IA. Die Standard-IA eignet sich für seltene Datenabrufe, die jedoch eine schnellere Zugriffszeit erfordern, wenn sie benötigt werden. Die Nearline-Klasse ähnelt der Standard-IA und ist ebenfalls für Daten gedacht, die gelegentlich im Zeitraum von Wochen oder Monaten abgerufen werden. Sie ist ideal für Datensicherung, Datenmigration und Datenarchivierung. Beide Speicherklassen bieten einen Kompromiss zwischen Kosteneinsparungen und Datenzugriffszeiten.\\

Während der Performance Analyse wurden diese beiden Speicherklassen verglichen, da sie ähnliche Eigenschaften aufweisen. Es stellte sich heraus, dass sich die Dauer des Uploads und Downloads ab 10 Dateien zu unterscheiden began. Dabei war die Standard-IA beim Hochladen und Herunterladen der zehn Dateien fast dreimal so schnell wie die Nearline-Klasse. Die Nearline-Klasse schnitt beim Hochladen schlechter ab. Beim Herunterladen von Dateien gab es jedoch nur minimale Unterschiede, wobei die Nearline bei 1000 Dateien 61.55\% langsamer war wie die Standard-IA.\\

Die Standardklassen beider Anbieter weisen kaum Unterschiede bei den PUT- und GET-Anfragen auf. Allerdings sind die Speicherungskosten in der Standardklasse von AWS 9.27\% höher als bei GC. Beide Klassen eignen sich für Daten, auf die häufig zugegriffen wird und die eine hohe Verfügbarkeit, schnelle Zugriffszeiten und geringe Latenzzeiten erfordern. Beide Standardklassen sind für den allgemeinen Gebrauch optimiert, wobei die Speicherungskosten in diesen Klassen im Vergleich zu anderen Speicherklassen am höchsten sind.\\

Bei der Performance-Analyse schnitt die Standardklasse von AWS ab zehn Dateien etwa 54.4\% besser ab als die entsprechende Klasse von GC. Beim Hochladen von 100 Dateien war die AWS-Standardklasse etwa 60.9\% schneller und bei 1000 Dateien etwa 58\% schneller wie die GC Standardklasse. Beim Herunterladen von Dateien waren die Unterschiede nicht groß genug, um eine eindeutige Aussage über die Überlegenheit einer Klasse zu treffen. Allerdings war die AWS-Standardklasse bei 1000 Dateien etwa 60\% schneller.\\

Es ist anzumerken, dass bei den AWS Kosten auch die Vorabzahlung der Speicherung aller Objekte enthalten ist. Bei GC gibt es hingegen keine Vorabzahlung für die Speicherung von Objekten. Aus diesem Grund und auch durch die niedrigeren Gebühren der Speicherung ist GC die kostengünstigere Datenspeicherung.\\

Insgesamt war die Dauer des Uploads und Downloads der Speicherklassen von AWS in der Performance Analyse niedriger als bei GC. Ab einer Datei traten bereits erste Unterschiede auf, wobei GC mehr Zeit in Anspruch nahm. Die Performance-Messungen basieren jedoch lediglich auf groben Schätzungen innerhalb einer virtuellen Umgebung, die von Hetzner Cloud bereitgestellt wurde. Hier ist anzumerken, dass der Hetzner Server näher am AWS Server liegen und deshalb bessere Ergebnisse in der Performance als GCP erzielen kann. Faktoren wie die Auslastung des Netzwerks können die Ergebnisse beeinflussen und stellen keine aussagekräftigen Performance-Ergebnisse dar.\\

\section{Bewertung des Prototyps}

Für die Bewertung des Prototyps wurden die Anforderungen von leoticket herangezogen. Dabei war es entscheidend, den Prototypen so zu bauen, dass die sichere Speicherung gedeckt war. Für die Deckung der sicheren Speicherung wurden Methoden betrachtet, die beide Cloud Provider anboten. Dabei implementiert der Prototyp die SSE-KMS Methode beider Provider. Durch die eigene Erstellung des Schlüssels im KMS von AWS und GC hat der Nutzer mehr Kontrolle, da die Schlüssel von ihm selbst erstellt werden. Der Schlüssel wird vom KMS gespeichert und muss daher nicht vom Nutzer extern gespeichert und verwaltet werden. Der Prototyp kann jedoch so umgebaut werden, dass auch eine andere Methode wie die SSE-C verwendet werden kann. Die SSE-C bietet eine höhere Sicherheit, da der Nutzer den Schüssel selbst generieren und speichern muss. Dies führt aber auch zum Risiko, den Schlüssel zu verlieren. In diesem Fall kann auf die Objekte im Bucket nicht mehr zugegriffen werden. Noch ein Kriterium von leoticket war die hohe Verfügbarkeit der Daten. Der Prototyp wurde so gebaut, dass der Nutzer die Speicherklasse für AWS selbst wählen kann. In GC funktioniert das, indem das Bucket die richtige Speicherklasse bereits eingestellt hat. Für die Verfügbarkeit sind jedoch die Cloud Provider verantwortlich. Hier versprechen beide Provider eine Verfügbarkeit von mindestens 99.5\%. Diese Verfügbarkeit hängt von den verschiedenen Speicherklassen ab. Über Terraform werden Buckets automatisch mit den Einstellungen konform zu den Anforderungen von leoticket erstellt. Diese beinhalten die Konfigurationen von:

\begin{itemize}
	\item Object Versioning
	\item Lifecycle Rules
	\item Object Ownership
	\item Data Encryption
	\item Object Logging
	\item Bucket ACL
	\item Public Access Block
\end{itemize}

Die Objekt Versionierung dient zur Steigerung der Verfügbarkeit, falls Daten unerwünscht gelöscht oder überschrieben werden. Für die Anforderung der Integration in Software-Produkte wie leoticket sorgen die SDKs der beiden Provider. Diese unterstützen mehrere Programmiersprachen. Sie werden durch Spring Boot unterstützt und können auch mit Maven oder Gradle gebaut werden. Die Anbindung verläuft gemäß der offiziellen Dokumentation der beiden Cloud Provider. Die Dokumentationen sind verständlich und auf dem aktuellsten Stand und beschreiben deren API. AWS und GC bieten auch neue Versionen an und updaten die SDKs. Die Generierung der signierten URLs ist für die Bereitstellung der Dateien zuständig. Diese Anforderung war das Hauptmerkmal von leoticket. Dabei ist es bei beiden Providern möglich, signierte, zeitlich begrenzte URLs zu generieren und diese Nutzern bereitzustellen. Empfänger der URL können so ihre Tickets und Rechnungen herunterladen. Für die Performance Analyse wird eine Methode zur Verfügung gestellt, die Testdateien in der Größe von 100 KB erstellen und in Buckets hoch- und herunterladen kann.\\

Insgesamt ist der Prototyp ausbaufähig und stellt für diese Arbeit einen Vergleich zwischen beiden Cloud Providern dar. Für den besseren Vergleich beider Cloud Provider wurden die Technologien ausprobiert und genutzt. Durch Änderungen an der Terraform Konfiguration, der Speicherklassen Variable für AWS und an der Methode der Datenverschlüsselung kann der Prototyp angepasst werden. Eine Schwäche des Prototyps besteht darin, dass keine genauen Performance Messungen durchgeführt werden können aufgrund genannter Einflüsse, welche die Ergebnisse beeinflussen können. Er dient lediglich dem groben Vergleich. Eine weitere Schwäche des Prototyps besteht darin, dass Entwickler bei der Integration des Prototyps in eigenen Anwendungen Anpassungen durchführen müssen, indem die Hauptklasse des Prototyps so umgebaut werden muss, dass sie dem gewünschten Verhalten entspricht.