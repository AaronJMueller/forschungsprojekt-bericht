\chapter{Einleitung}
\autor{Medjen Izairi}
Die Entwicklung autonomer Fahrzeugtechnologien hat in den letzten Jahren eine transformative Phase in der Mobilitätsbranche eingeleitet. Autonomes Fahren, auch als \enquote{self-driving} oder \enquote{driverless} bezeichnet, bezieht sich auf Fahrzeuge, die in der Lage sind, ohne menschliches Eingreifen zu navigieren und zu operieren. \cite{m1} Mit der fortschreitenden Integration autonomer Fahrzeugtechnologien in unsere Gesellschaft gewinnt die Förderung von Innovation und Fachwissen auf diesem Gebiet an entscheidender Bedeutung. Ein herausragendes Beispiel, das Studierende weltweit in diese Entwicklungen einbindet, ist die \gls{BFMC}. 
Die \gls{BFMC} ist ein internationaler technischer Wettbewerb, der 2017 vom Bosch Engineering Center Cluj, Rumänien, initiiert wurde. Der Wettbewerb lädt jedes Jahr Studententeams ein, autonome Fahralgorithmen für Fahrzeuge zu entwickeln, die vom Unternehmen zur Verfügung gestellt werden. Die Studierenden sollen mehrere Monate lang in Zusammenarbeit mit Bosch-Experten und Hochschulprofessoren an ihren Projekten arbeiten, um diese Algorithmen zu entwickeln. \cite{boschfumobility} Teilnahmeberechtigt sind Teams aus aller Welt, und es wird keine Unterscheidung zwischen früheren Teilnehmern und Neulingen gemacht. Jedes Team setzt sich aus drei bis fünf Studierenden zusammen, einschließlich des Teamleiters, sowie einem Mentor aus dem akademischen Bereich. Die Mitglieder können aus verschiedenen Studienrichtungen und Universitäten stammen. 
Während des gesamten Wettbewerbs verpflichten sich die Teams zur regelmäßigen Einreichung von monatlichen Projektstatus-Updates über die Wettbewerbswebsite. Diese Updates bestehen aus technischen Berichten, Projektplänen und Videos, die den Fortschritt des Teams zeigen. Die Qualität dieser Einreichungen trägt zur Gesamtbewertung bei. 
Die Qualifikationsphase fungiert als Ausscheidungsrunde, in der die Teams ihre Fähigkeit zur Entwicklung einer funktionalen Lösung demonstrieren müssen. Jedes Team reicht ein Video ein, das das autonom fahrende Fahrzeug bei einer Sequenz von Aktionen zeigt. Nur die besten 24 aus 80 Teams erreichen die nächste Phase als „The Competition Days“ bezeichnet. 
Die qualifizierten Teams erhalten die Gelegenheit, ein Werbevideo nach spezifischem Inhalt und Vorlage zu erstellen, das dann auf der Wettbewerbswebsite präsentiert wird.  Diese Teams werden darüber hinaus nach Cluj-Napoca, Rumänien, eingeladen, um die Ergebnisse ihrer Arbeit stolz vor einer internationalen Jury zu präsentieren. Am Ende werden die herausragendsten Lösungen mit beeindruckenden Preisen von Bosch belohnt.\cite{bfmc} 