\chapter{Einleitung}

Das folgende Kapitel dient der Einführung in die Problemstellung, Motivation sowie Ziele und Vorgehensweisen der vorliegenden Arbeit.

\section{Problemstellung und Motivation}

Die steigende Menge an Daten im Kontext von Web Services, die in verschiedenen Anwendungen generiert werden, stellt eine große Herausforderung dar. Diese Bachelorarbeit richtet sich auf die Herausforderung einer Full-Service-Ticketing Software „leoticket“, die vom Unternehmen Leomedia GmbH entwickelt wird. 
Leomedia GmbH ist ein Unternehmen, das Software für Medienunternehmen wie Zeitungsverlage, Radiosender, Veranstalter, Künstler und Kulturvereine entwickelt.\footcite{leomedia-web} 
Leoticket ist eines der vielen Produkte von Leomedia, dass Services wie Online-Kartenvorverkäufe, Abendkassen, den Einlass bei der Veranstaltung, Statistiken, Abrechnungen und die Planung der Veranstaltung realisiert.\footcite{leomedia-web}\\ 

Die erfassten Daten umfassen verschiedene Arten von Dokumenten, wie beispielsweise PDF-Dateien, insbesondere Tickets und Rechnungen, die aus dem Kaufprozess über die leoticket Platform resultieren. Dabei ist es von großer Bedeutung, dass diese Daten von leoticket-Kunden und Leomedia sicher, zuverlässig und schnell gespeichert und abgerufen werden können. Vor diesem Hintergrund stellen sich Fragen nach der Auswahl eines geeigneten Speichersystems, welches die Performance, Verfügbarkeit, Sicherheitsanforderungen und die Möglichkeit der Integration in Software-Produkten wie leoticket erfüllt. Zudem müssen Mechanismen bereitgestellt werden, um den Zugriff der Ticketbesitzer auf die Daten durch sichere, zeitlich begrenzte URL’s zu beschränken.\\

Die Herausforderung von leoticket betrifft die Speicherung und Bereitstellung von Daten wie Tickets und Rechnungen an die Ticketkäufer. Ein Galera Cluster ist eine Multi-Master-Replikationslösung für relationale Datenbanken. Es basiert auf dem Konzept der synchronen Replikation, bei dem mehrere Knoten zu einem Cluster miteinander verbunden. 
Im Rahmen des Replikationsprozesses werden Daten synchronisiert, wodurch sich die Leistung bei Anfragen verringert, da das Datenbanksystem durch die Ausführung des Replikations- bzw. Synchronisierungsjobs beansprucht wird. Dabei erreicht der Arbeitsspeicher seine Kapazitätsgrenze von 200 GB. Die Hauptaufgaben der Datenbank umfassen beispielsweise die Durchführung von JOINS.\\ 

Eine weitere Herausforderung ist die Bereitstellung der Dateien über Email Anhänge. Anhänge dürfen eine bestimmte Speichergröße nicht überschreiten. Wenn Ticketkäufer mehrere Tickets in einer Bestellung tätigen, dann müssen diese über Email Anhänge bereitgestellt werden.\\

Leomedia plant eine Neugestaltung der leoticket-Anwendung, bei der sie sich von der Galera Cluster Technologie lösen möchten. In dieser Untersuchung werden keine relationalen Datenbanken berücksichtigt. Relationale Datenbanksysteme basieren auf einem festen Schema, das vorab definiert werden muss. Dies kann problematisch sein, wenn die Struktur der erfassten Daten des Produkts leoticket häufig geändert oder erweitert werden muss. NoSQL-Datenbanken bieten in dieser Hinsicht oft mehr Flexibilität, da sie schemalos oder mit flexiblen Schemas arbeiten. Weitere Gründe sind die Skalierbarkeit, Komplexität und der Speicherplatzbedarf. Es ist wichtig anzumerken, dass diese Gründe nicht bedeuten, dass relationale Datenbanksysteme grundsätzlich ausgeschlossen werden sollten. Sie sind nach wie vor eine bewährte und weit verbreitete Lösung für viele Anwendungen. Die Entscheidung, ein relationales Datenbanksystem auszuschließen, hängt von den spezifischen Anforderungen und Herausforderungen ab, die vorab für diese Arbeit definiert wurden.

\newpage

\section{Zieldefinition und Vorgehensweise}

Das Ziel dieser Arbeit besteht darin, anhand der Anforderungen von leoticket eine geeignete Speicherlösung zu empfehlen und die Realisierung eines Prototypen basierend auf den ausgewählten Cloud-Providern zu erstellen. Der Prototyp soll die Bereitstellung vordefinierter Daten durch sichere und zeitlich begrenzte URLs ermöglichen. 

Dabei werden folgende Fragen gestellt:

\begin{itemize}
	\item Welches Speichersystem ist im Hinblick auf Kosten, Performance und Verfügbarkeit für die Persistenz von Daten besonders geeignet? 
	\item Wie können diese Daten durch sichere, zeitlich begrenzte URL's bereitgestellt werden?
\end{itemize}

Zur Beantwortung werden verschiedene Arten von Speichersystemen untersucht. Dabei erfolgt eine Analyse  aktuell verfügbarer Speichertechnologien hinsichtlich ihrer nicht-funktionalen Eigenschaften wie Sicherheit, Verfügbarkeit, Performance und Kosten, die im Zusammenhang mit den verwendeten Technologien entstehen können. Aus funktionaler Sicht wird auch die Möglichkeit der Integration des Speichersystems in Software Produkten und die sichere Bereitstellung der erfassten Dateien betrachtet. Cloud-Provider werden miteinander verglichen und Kosten-, und Performance-Kalkulationen durchgeführt. Die Ergebnisse werden anschließend ausgewertet und interpretiert, um eine geeignete Speicherlösung zu empfehlen.\\

Der Prototyp wird auf Basis der von Cloud Providern angebotenen ausgewählten Technologien und Empfehlung implementiert. Nach der Durchführung von Messungen zur Performance auf Testdaten erfolgt eine Zusammenfassung der Implementierung.\\

Zum Abschluss werden die Ergebnisse präsentiert, interpretiert und eine Bewertung des Prototyps vorgenommen.  Die am Anfang gestellten Fragen der Arbeit werden beantwortet und potenzielle Anwendungen des Prototyps aufgelistet.

 