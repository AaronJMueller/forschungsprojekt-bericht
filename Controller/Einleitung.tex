\section{Einleitung und Motivation}
\autor{Aaron J. Müller}
Damit ein autonomes Fahrzeug zuverlässig einer berechneten Solltrajektorie folgen kann, ist ein robuster und effizienter Regler notwendig.
Häufig wird dieser in zwei Teilprobleme aufgeteilt: Ein Longitudinalregler regelt die Fahrgeschwindigkeit, während ein Lateralregler die Lenkung des Fahrzeugs steuert. 
In diesem Kapitel soll die Entwicklung eines solchen Lateralreglers dokumentiert werden, welcher den bestehenden Lateralregler ersetzen soll.
Zur Auswahl stehen dabei zwei grundlegend unterschiedliche Reglertypen; Stanley und \acs{MPC}. 
Beide sollen verglichen werden, um auf dieser Basis zu entscheiden welche Art von Regler implementiert werden soll.

Unabhängig von der Wahl des Reglers ist folgendes Vorgehen vorgesehen: 
\begin{itemize}
    \item Konzeption des Reglers, basierend auf Modellen des Fahrzeugs sowie der Fahrphysik
    \item Implementierung des Reglers in einer MATLAB/Simulink Umgebung
    \item Integration in den it:movES Softwarestack und Erprobung im Unity-Simulator
    \item Inbetriebnahme auf dem realen Fahrzeug und Feintuning
\end{itemize}

Dadurch wird sowohl die theoretische Güte des Regelverhaltens als auch die tatsächliche Performance im Gesamtsystem ermittelt.
Das ist notwendig, da die Solltrajektorie im Realbetrieb eine schlechtere Qualität als die simulierte hat, was das Regelverhalten maßgeblich beeinflusst.