\chapter{Projekt Überblick}

\section{Motivation}
\autor{Medjen Izairi}
In mehrmonatiger Zusammenarbeit zwischen den Bosch-Experten und Hochschulprofessoren und Studenten soll eine zuverlässige Lösung für die Kernfunktionen eines autonom fahrenden Fahrzeugs im Maßstab 1:10 entwickelt werden. Diese Kernfunktionen umfassen in einer Miniatur-Smart-City unterschiedliche Aufgaben. Dazu gehören Spurhaltung, sicheres Navigieren in Kreuzungen, adäquate Reaktionen auf Verkehrslichter, präzise Navigation basierend auf Lokalisierungsdaten und die Fähigkeit zur sicheren Interaktion mit anderen Verkehrsteilnehmern.\cite{boschfumobility}
Zu den herausfordernden Aufgaben gehört unter anderem die präzise Spurhaltung, um eine sichere und stabile Fahrweise zu gewährleisten. Ebenso wird die Fähigkeit des Fahrzeugs, sich sicher durch Kreuzungen zu navigieren, intensiv getestet und weiterentwickelt. Dabei stehen adäquate Reaktionen auf Verkehrslichter im Fokus, um eine reibungslose Interaktion mit der Umgebung zu gewährleisten. 
Des Weiteren legt das Projekt einen Schwerpunkt auf die präzise Navigation, basierend auf hochgenauen Lokalisierungsdaten. Die Entwicklung einer robusten Lokalisierungstechnologie ist entscheidend, um das Fahrzeug in der Miniatur-Smart-City präzise und zuverlässig zu positionieren. Gleichzeitig wird die Fähigkeit des Fahrzeugs zur sicheren Interaktion mit anderen Verkehrsteilnehmern, wie Fußgängern und anderen Fahrzeugen, eingehend erforscht und optimiert.

\newpage

 
\section{Zieldefinition und Vorgehensweise}
\autor{Medjen Izairi}
Die Herangehensweise an die Projektentwicklung gliedert sich in vier Hauptdomänen: Perception, Sensing, Planning und Controlling. Jede Domäne umfasst spezifische Aufgaben, die in vier Unterphasen durchgeführt werden: Vorbereitungsphase, Planungsphase, Implementierungsphase und Testphase. Die klare Struktur erstreckt sich von der Einarbeitung und Forschung über die Planung und Umsetzung bis hin zu umfassenden Tests. Zusätzlich zur Softwareentwicklung werden Hardware-Aspekte berücksichtigt, darunter Autolieferungen und die Entscheidung für zusätzliche Hardwareteile. Die Lieferung und der Einbau dieser Teile sind entscheidende Schritte, um sicherzustellen, dass die entwickelten Technologien sowohl auf der Software- als auch auf der Hardwareebene effektiv zusammenarbeiten. Das gemeinsame Ziel besteht darin, die Miniatur-Smart-City zu einem realistischen Testumfeld zu machen, in dem die entwickelten Technologien ihre Wirksamkeit unter realitätsnahen Bedingungen unter Beweis stellen können.


\newpage

\section{Aufbau der Arbeit}

\autor{Aaron Müller}

Nach dem Projektüberblick beschäftigt sich das dritte Kapitel mit dem Gesamtprojekt.
Es werden die Rahmenbedingungen der \gls{BFMC} 2024 erläutert, insbesondere Veränderungen zu vergangengen Wettbewerben. 
Außerdem wird der it:movES Softwarestack erläutert, auf dem diese Arbeit aufbaut, und es werden Hardware-Upgrades am bereitgestellten Fahrzeug diskutiert.

Die folgenden Kapitel Lateralregler und Behavior Tree befassen sich mit den großen Aufgabenfeldern, die den Kernpunkt dieser Arbeit bilden. 
Diese Kapitel sind in sich wie eine klassische Forschungsarbeit gegliedert, und können so auch unabhängig voneinander betrachtet werden.

Das letzte Kapitel schließt die Arbeit ab, in dem ein Fazit zum Gesamtprojekt gezogen wird.
Dabei werden die Ergebnisse diskutiert und Ausblicke für weitere Forschungsarbeiten im Kontext des it:movES Softwarestacks gegeben.